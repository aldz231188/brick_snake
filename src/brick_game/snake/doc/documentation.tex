\documentclass{article}

\usepackage{fontspec}
\setmainfont{Liberation Serif}
\usepackage{polyglossia}
\setmainlanguage{russian}

\title{BrickGame v2.0}
\author{medgarna}
\date{ноябрь 2024}

\begin{document}

\maketitle

\section{Тетрис}
\subsection{Описание}
{\small Программа разработана на языке Си с использованием компилятора gcc. Консольный интерфейс разработан с помощью библиотеки \textbf{ncurses}. 
Игровое поле — десять «пикселей» в ширину и двадцать «пикселей» в высоту.}

\subsection{Управление}
Для управления поддерживаются следующие кнопки на физической консоли:
\begin{itemize}
    \item 'S' — Начало игры,
    \item 'Пробел' — Пауза,
    \item 'Q' — Завершение игры,
    \item Стрелка влево — движение фигуры влево,
    \item Стрелка вправо — движение фигуры вправо,
    \item Стрелка вниз — падение фигуры,
    \item Стрелка вверх — не используется в данной игре,
    \item 'A' — Действие (вращение фигуры).
\end{itemize}

\subsection{Конечный автомат}
КА тетриса состоит из следующих состояний:
\begin{itemize}
    \item \textbf{START} — состояние, в котором игра ждёт, пока игрок нажмёт кнопку готовности к игре ('S').
    \item \textbf{PAUSE} — состояние, в котором игра ждёт, пока игрок нажмёт кнопку ('Пробел').
    \item \textbf{SPAWN} — состояние, в которое переходит игра при создании очередного блока и выбора следующего блока для спавна.
    \item \textbf{GAME} — основное игровое состояние с обработкой ввода от пользователя — поворот блоков, перемещение блоков по горизонтали и падение блоков.
    \item \textbf{ATTACH} — состояние, в которое переходит игра после соприкосновения текущего блока с уже упавшими или с землёй. Если образуются заполненные линии, то они уничтожаются, остальные блоки смещаются вниз, и игра переходит в состояние SPAWN. Если блок остановился в самом верхнем ряду, то игра переходит в состояние START.
    \item \textbf{GAMEOVER} — состояние, в которое переходит игра в случае проигрыша. После нажатия кнопки ('S') игра переходит в состояние START.
    \item \textbf{EXITSTATE} — если игрок нажимает на кнопку 'Q', то игра завершается.
\end{itemize}

\section{Змейка}
\subsection{Описание}
{\small Программа разработана на языке программирования С++ в парадигме объектно-ориентированного программирования. Консольный интерфейс разработан с помощью библиотеки \textbf{ncurses}. 
Десктопный интерфейс разработан на базе GUI-библиотеки Qt.
Игровое поле — десять «пикселей» в ширину и двадцать «пикселей» в высоту.}

\subsection{Управление}
Для управления поддерживаются следующие кнопки на физической консоли:
\begin{itemize}
    \item 'S' — Начало игры,
    \item 'Пробел' — Пауза,
    \item 'Q' — Завершение игры,
    \item Стрелка влево — изменить направление движения,
    \item Стрелка вправо — изменить направление движения,
    \item Стрелка вниз — изменить направление движения,
    \item Стрелка вверх — изменить направление движения,
    \item 'A' — Действие (ускорение движения змейки).
\end{itemize}

\subsection{Конечный автомат}
КА змейки состоит из следующих состояний:
\begin{itemize}
    \item \textbf{Start\_State} — состояние, в котором игра ждёт, пока игрок нажмёт кнопку готовности к игре ('S').
    \item \textbf{Listen\_State} — состояние, в котором игра ждёт, пока игрок нажмёт любую кнопку.
    \item \textbf{Play\_State} — основное состояние, в котором происходит движение змеи по истечению таймера.
    \item \textbf{Pause\_State} — состояние, в котором игра ждёт следующего нажатия клавиши 'Пробел'.
    \item \textbf{GameOver\_State} — состояние, в которое переходит игра после столкновения со стеной или самой с собой.
    \item \textbf{Terminate\_State} — если игрок нажимает на кнопку 'Q', то игра завершается.
\end{itemize}

\end{document}
